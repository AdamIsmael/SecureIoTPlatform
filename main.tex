%%%%%%%%%%%%%%%%%%%%%%%%%%%%%%%%%%%%%%%%%
% Masters/Doctoral Thesis 
% LaTeX Template
% Version 2.2 (21/11/15)
%
% This template has been downloaded from:
% http://www.LaTeXTemplates.com
%
% Version 2.x major modifications by:
% Vel (vel@latextemplates.com)
%
% This template is based on a template by:
% Steve Gunn (http://users.ecs.soton.ac.uk/srg/softwaretools/document/templates/)
% Sunil Patel (http://www.sunilpatel.co.uk/thesis-template/)
%
% Template license:
% CC BY-NC-SA 3.0 (http://creativecommons.org/licenses/by-nc-sa/3.0/)
%
%%%%%%%%%%%%%%%%%%%%%%%%%%%%%%%%%%%%%%%%%



%----------------------------------------------------------------------------------------
%	PACKAGES AND OTHER DOCUMENT CONFIGURATIONS
%----------------------------------------------------------------------------------------

\documentclass[
11pt, % The default document font size, options: 10pt, 11pt, 12pt
%oneside, % Two side (alternating margins) for binding by default, uncomment to switch to one side
english, % ngerman for German
singlespacing, % Single line spacing, alternatives: onehalfspacing or doublespacing
%draft, % Uncomment to enable draft mode (no pictures, no links, overfull hboxes indicated)
%nolistspacing, % If the document is onehalfspacing or doublespacing, uncomment this to set spacing in lists to single
%liststotoc, % Uncomment to add the list of figures/tables/etc to the table of contents
%toctotoc, % Uncomment to add the main table of contents to the table of contents
%parskip, % Uncomment to add space between paragraphs
%nohyperref, % Uncomment to not load the hyperref package
headsepline, % Uncomment to get a line under the header
]{MastersDoctoralThesis} % The class file specifying the document structure

\usepackage[utf8]{inputenc} % Required for inputting international characters
\usepackage[T1]{fontenc} % Output font encoding for international characters

\usepackage{palatino} % Use the Palatino font by default

%autocite = superscript    maybe without natbib=true
\usepackage[backend=bibtex,natbib=true,style=numeric,sorting=none]{biblatex} % User the bibtex backend with the authoryear citation style (which resembles APA)
%\usepackage[style=authoryear,natbib=true]{biblatex}
\addbibresource{example.bib} % The filename of the bibliography
\bibliography{example}

%\usepackage[autostyle=true]{csquotes} % Required to generate language-dependent quotes in the bibliography

\usepackage[dvipsnames]{xcolor} 
% Have to load this before tikz as tikz wants a normal version of xcolor without the dvipsnames option but if the package with special options
% is loaded first then that seems to be okay

%----------------------------------------------------------------------------------------
%	TIKZ BLOCK DIAGRAM SETTINGS
%----------------------------------------------------------------------------------------

\usepackage{tikz}
\usetikzlibrary{shapes,arrows}

\tikzstyle{block} = [draw, fill=blue!20, rectangle, 
    minimum height=3em, minimum width=6em]
\tikzstyle{sum} = [draw, fill=blue!20, circle, node distance=1cm]
\tikzstyle{input} = [coordinate]
\tikzstyle{output} = [coordinate]
\tikzstyle{pinstyle} = [pin edge={to-,thin,black}]

%----------------------------------------------------------------------------------------
%    CODE LISTING SETTINGS
%----------------------------------------------------------------------------------------

\usepackage{float}
\usepackage{listings}
\usepackage{color}


\usepackage{typearea}
\usepackage[utf8]{inputenc}
\usepackage[T1]{fontenc}
\usepackage{mwe} 
\usepackage{etoolbox}  

\definecolor{dkgreen}{rgb}{0,0.6,0}
\definecolor{grey}{rgb}{0.5,0.5,0.5}
\definecolor{mauve}{rgb}{0.58,0,0.82}
\definecolor{myorange}{rgb}{0.8,0.4,0}
\definecolor{myblue}{rgb}{0.1,0.61,0.98}
\definecolor{darkblue}{rgb}{.5,0.61,0.98}
\definecolor{lightbrown}{rgb}{0.6, 0.38, 0.23}
\definecolor{beaver}{rgb}{0.62, 0.51, 0.44}
\lstset{frame=tb,
  language=C,
  aboveskip=3mm,
  belowskip=3mm,
  showstringspaces=false,
  columns=flexible,
  basicstyle={\small\ttfamily},
  morekeywords={HEX}
  numbers=left,
  numbersep=5pt,                   % how far the line-numbers are from the code
  numberstyle=\tiny\color{grey},
  keywordstyle=\color{red},
  commentstyle=\color{dkgreen},
  breaklines=true,
  breakatwhitespace=true,
  tabsize=3,
}



\lstdefinestyle{Arduino}{%
    %style=FormattedNumber,
    keywords={println, print, for, OneWire,select,write,delay, reset, byte, char, const, long, unsigned, stop, available, read, if, else,int, String},%                 define keywords
    morecomment=[l]{//},%             treat // as comments
    morecomment=[s]{/*}{*/},%         define /* ... */ comments
    emph={HIGH, OUTPUT, LOW, setup, loop, Serial},%        keywords to emphasize
    numbers=left,
    keywordstyle=\color{myorange},
   emphstyle=\bfseries\color{myorange},
  stringstyle=\color{myblue},
}

\lstdefinestyle{PHP}{
  %style=FormattedNumber,
    keywords={php},%                 define keywords
    morecomment=[l]{//},%             treat // as comments
    morecomment=[s]{/*}{*/},%         define /* ... */ comments
    emph={include, mysql_query, mysql_close},%        keywords to emphasize
    numbers=left,
    emphstyle=\bfseries\color{NavyBlue},
    keywordstyle=\color{red},
    stringstyle=\color{grey},
}

\lstdefinestyle{SQL}{
  %style=FormattedNumber,
    keywords={CREATE, TABLE, INT, NOT, NULL, AUTO_INCREMENT, TIMESTAMP, DEFAULT, CURRENT_TIMESTAMP, VARCHAR, PRIMARY, KEY, ENGINE},%                 define keywords
    morecomment=[l]{//},%             treat // as comments
    morecomment=[s]{/*}{*/},%         define /* ... */ comments
    emph={},%        keywords to emphasize
    emphstyle=\bfseries\color{NavyBlue},
    keywordstyle=\color{blue},
    stringstyle=\color{grey},
}

\lstdefinestyle{Java}{
  %style=FormattedNumber,
    keywords={request, response,printWriter, signedCipherArray, ArduinoPublicSignatureKey, javaPlainTextMessage, signedMessage, cipher, messageLength, nonce, arduinoPublicKey, serverSecretKey, DB_URL, user, password, sql, rs, id, tempHex, timeStamp, conn, stmt},%                 define keywords
    morecomment=[l]{//},%             treat // as comments
    morecomment=[s]{/*}{*/},%         define /* ... */ comments
    emph={protected, void, throws, byte, final, try},%        keywords to emphasize
    emphstyle=\bfseries\color{mauve},
    keywordstyle=\color{lightbrown},
    stringstyle=\color{blue},
    numbers=left,
}

%----------------------------------------------------------------------------------------
%	MARGIN SETTINGS
%----------------------------------------------------------------------------------------

\geometry{
	paper=a4paper, % Change to letterpaper for US letter
	inner=2.5cm, % Inner margin
	outer=3.8cm, % Outer margin
	bindingoffset=2cm, % Binding offset
	top=1.5cm, % Top margin
	bottom=1.5cm, % Bottom margin
	%showframe,% show how the type block is set on the page
}

%----------------------------------------------------------------------------------------
%	THESIS INFORMATION
%----------------------------------------------------------------------------------------

\thesistitle{Secure Internet Of Things Sensor Platform} % Your thesis title, this is used in the title and abstract, print it elsewhere with \ttitle
\supervisor{Dr. James Irvine} % Your supervisor's name, this is used in the title page, print it elsewhere with \supname
\examiner{Dr Alex Coddington} % Your examiner's name, this is not currently used anywhere in the template, print it elsewhere with \examname
\degree{Computer and Electronic Systems} % Your degree name, this is used in the title page and abstract, print it elsewhere with \degreename
\author{Adam Kidd} % Your name, this is used in the title page and abstract, print it elsewhere with \authorname
\addresses{} % Your address, this is not currently used anywhere in the template, print it elsewhere with \addressname
\phdstudent{Grieg Paul}

\subject{IoT} % Your subject area, this is not currently used anywhere in the template, print it elsewhere with \subjectname
\keywords{} % Keywords for your thesis, this is not currently used anywhere in the template, print it elsewhere with \keywordnames
\university{\href{http://www.strath.ac.uk}{University Of Strathclyde}} % Your university's name and URL, this is used in the title page and abstract, print it elsewhere with \univname
\department{\href{http://www.strath.ac.uk/engineering/electronicelectricalengineering/}{Department of Electronic \& Electrical Engineering}} % Your department's name and URL, this is used in the title page and abstract, print it elsewhere with \deptname
%\group{\href{http://researchgroup.university.com}{Research Group Name}} % Your research group's name and URL, this is used in the title page, print it elsewhere with \groupname
\faculty{\href{http://www.strath.ac.uk/science/}{Faculty Of Science}} % Your faculty's name and URL, this is used in the title page and abstract, print it elsewhere with \facname

\hypersetup{pdftitle=\ttitle} % Set the PDF's title to your title
\hypersetup{pdfauthor=\authorname} % Set the PDF's author to your name
\hypersetup{pdfkeywords=\keywordnames} % Set the PDF's keywords to your keywords

\begin{document}

\frontmatter % Use roman page numbering style (i, ii, iii, iv...) for the pre-content pages

\pagestyle{plain} % Default to the plain heading style until the thesis style is called for the body content

%----------------------------------------------------------------------------------------
%	TITLE PAGE
%----------------------------------------------------------------------------------------

\begin{titlepage}
\begin{center}

\textsc{\LARGE \univname}\\[1.5cm] % University name
\textsc{\Large Technical Report}\\[0.5cm] % Thesis type

\HRule \\[0.4cm] % Horizontal line
{\huge \bfseries \ttitle}\\[0.4cm] % Thesis title
\HRule \\[1.5cm] % Horizontal line
 
\begin{minipage}{0.4\textwidth}
\begin{flushleft} \large
\emph{Author:}\\
\href{http://www.johnsmith.com}{\authorname} \\ % Author name - remove the \href bracket to remove the link
\emph{Examiner:}\\
\examname
\end{flushleft}
\end{minipage}
\begin{minipage}{0.4\textwidth}
\begin{flushright} \large
\emph{Supervisor:} \\
\href{http://www.jamessmith.com}{\supname}\\ % Supervisor name - remove the \href bracket to remove the link  
\emph{Phd Student:}\\
\phdname
\end{flushright}
\end{minipage}\\[3cm]
 
\large \textit{A technical report submitted in fulfillment of the requirements\\ for the degree of \degreename}\\[0.3cm] % University requirement text
\textit{in the}\\[0.4cm]
\deptname\\[1cm] % Research group name and department name



 {\large \today}\\[1cm] % Date
 \bigskip
\includegraphics{img/uni_logo_eng.jpg} % University/department logo - uncomment to place it
 
\vfill
\end{center}
\end{titlepage}

%----------------------------------------------------------------------------------------
%	DECLARATION PAGE
%----------------------------------------------------------------------------------------

\begin{declaration}
\addchaptertocentry{\authorshipname}

\bigskip

\noindent I, \authorname, “I hereby declare that this work titled, \enquote{\ttitle} has not been submitted for any other degree/course
at this University or any other institution and that, except where reference is made
to the work of other authors, the material presented is original and entirely the
result of my own work at the University of Strathclyde under the supervision of ..
insert supervisor’s name.”
\\

%\begin{itemize} 
%\item This work was done wholly or mainly while in candidature for a research degree at this University.
%\item Where any part of this thesis has previously been submitted for a degree or any other qualification at this University or any other institution, this has been clearly stated.
%\item Where I have consulted the published work of others, this is always clearly attributed.
%\item Where I have quoted from the work of others, the source is always given. With the exception of such quotations, this thesis is entirely my own work.
%\item I have acknowledged all main sources of help.
%\item Where the thesis is based on work done by myself jointly with others, I have made clear exactly what was done by others and what I have contributed myself.\\
%\end{itemize}
 
\noindent Signed:\\
\rule[0.5em]{25em}{0.5pt} % This prints a line for the signature
 
\noindent Date:\\
\rule[0.5em]{25em}{0.5pt} % This prints a line to write the date
\end{declaration}

\cleardoublepage

%----------------------------------------------------------------------------------------
%	QUOTATION PAGE
%----------------------------------------------------------------------------------------

%\vspace*{0.2\textheight}

%\noindent\enquote{\itshape Thanks to my solid academic training, today I can write hundreds of words on virtually any topic without possessing a shred of %information, which is how I got a good job in journalism.}\bigbreak

%\hfill Dave Barry

%----------------------------------------------------------------------------------------
%	ABSTRACT PAGE
%----------------------------------------------------------------------------------------

\begin{abstract}
\addchaptertocentry{\abstractname} % Add the abstract to the table of contents

This report aims to explain how to build a sensor platform that can transmit data across the internet securely and show that security on low powered microcontrollers is possible and easy. It introduces a light weight cryptographic library and demonstrates it's effectiveness on a low powered system and provides considerations on the level of security it provides.

\end{abstract}

%----------------------------------------------------------------------------------------
%	ACKNOWLEDGEMENTS
%----------------------------------------------------------------------------------------

\begin{acknowledgements}
\addchaptertocentry{\acknowledgementname} % Add the acknowledgements to the table of contents

Over the course of this project I was helped on more than one occasion. I would like to give thanks to my project supervisor, Dr. James Irvine and Grieg Paul, a PhD student who help was very useful throughout.

\end{acknowledgements}

%----------------------------------------------------------------------------------------
%	LIST OF CONTENTS/FIGURES/TABLES PAGES
%----------------------------------------------------------------------------------------

\tableofcontents % Prints the main table of contents

\listoffigures % Prints the list of figures

%\listoftables % Prints the list of tables

%----------------------------------------------------------------------------------------
%	ABBREVIATIONS
%----------------------------------------------------------------------------------------

%\begin{abbreviations}{ll} % Include a list of abbreviations (a table of two columns)

%\textbf{LAH} & \textbf{L}ist \textbf{A}bbreviations \textbf{H}ere\\
%\textbf{WSF} & \textbf{W}hat (it) \textbf{S}tands \textbf{F}or\\

%\end{abbreviations}

%----------------------------------------------------------------------------------------
%	PHYSICAL CONSTANTS/OTHER DEFINITIONS
%----------------------------------------------------------------------------------------

%\begin{constants}{lr@{${}={}$}l} % The list of physical constants is a three column table

% The \SI{}{} command is provided by the siunitx package, see its documentation for instructions on how to use it

%	Speed of Light & $c_{0}$ & \SI{2.99792458e8}{\meter\per\second} (exact)\\
%Constant Name & $Symbol$ & $Constant Value$ with units\\

%\end{constants}

%----------------------------------------------------------------------------------------
%	SYMBOLS
%----------------------------------------------------------------------------------------

%\begin{symbols}{lll} % Include a list of Symbols (a three column table)

%$a$ & distance & \si{\meter} \\
%$P$ & power & \si{\watt} (\si{\joule\per\second}) \\
%Symbol & Name & Unit \\

%\addlinespace % Gap to separate the Roman symbols from the Greek

%$\omega$ & angular frequency & \si{\radian} \\

%\end{symbols}

%----------------------------------------------------------------------------------------
%	DEDICATION
%----------------------------------------------------------------------------------------

%\dedicatory{For/Dedicated to/To my\ldots} 

%----------------------------------------------------------------------------------------
%	THESIS CONTENT - CHAPTERS
%----------------------------------------------------------------------------------------

\mainmatter % Begin numeric (1,2,3...) page numbering

\pagestyle{thesis} % Return the page headers back to the "thesis" style

% Include the chapters of the thesis as separate files from the Chapters folder
% Uncomment the lines as you write the chapters

%Write at a technical level such that a pro engineer in another field can understand it
%and so that another student could follow and recreate what I build.
%So, assume general competence but explain new information


%\include{Chapters/Chapter1}
%Chapter of Introductions
%Pretty self explanatory really

\chapter{Introduction}
\label{intro}

%----------------------------------------------------------------------------------------

% Define some commands to keep the formatting separated from the content 
\newcommand{\keyword}[1]{\textbf{#1}}
\newcommand{\tabhead}[1]{\textbf{#1}}
\newcommand{\code}[1]{\texttt{#1}}
\newcommand{\file}[1]{\texttt{\bfseries#1}}
\newcommand{\option}[1]{\texttt{\itshape#1}}

%----------------------------------------------------------------------------------------

The Internet of Things or IoT is the concept of a huge network of physical objects connected together and communicating between themselves and to the world wide web. Devices can include domestic appliances, cars and even buildings. It is a rapidly growing field with over 50 million devices expected to be connected to the web by 2020\cite{50milby2020}. As such the security of the transmissions of these devices is becoming a more and more pressing issue. IoT's main benefits are the remote control of devices and appliances, the ability of the device to send information about it's state, such as a vending machine reporting that it has run out of a certain item, and to allow the machines to be more automated and to work with other machines, like a home hub device that can turn on the lights and central heating when an occupant is arriving home, with the lights and heating not being connected to each other but to the central hub.
	
	However IoT benefits will be severely limited if it unsecure. IoT is an emerging field but there have already been some high profile security disasters. Ranging from relatively less serious problems such as some attackers being able to glean WiFi network information from internet connected lights\cite{hackingsum} to the very concerning and potentially fatal security breaches like someone gaining unauthorised access to your car and assuming control. There have been three examples of this in the last few years with a Jeep Cherokee\cite{jeephack}, a Toyota Prius\cite{priushack} and a Tesla Model S\cite{teslahack} being the cars effected. The hackers were able to control the accelerator, door locks and brakes, among other things. In the Jeep Cherokee the attackers remotely disabled the engine of the car as it was driving up a freeway at 70 MPH. This highlights a very real problem that will only become more important as the field grows. Too often security is an afterthought but it really needs to be built into products from the offset.

	The challenge is to provide a cryptographic solution that is similar in strength of security to solutions that are implemented on more powerful servers and computers but on much smaller and less powerful devices. A solution that performs at an acceptable speed and provides good security with reduced power consumption.
	
	Google and British Gas have recently released Nest and Hive respectively. Nest was released in 2014 and Hive in 2013. These both involve controlling your central heating remotely and programming in days when you won't be at home and therefore have no need of heating. However, it was two whole years later upon independent investigators discovering that information, such as the dates when the heating was on or off, was being sent unencrypted that British Gas decided to encrypt their product Hive's transmissions. This plaintext data could be used to work out when the occupants were away as the central heating was likely to be off in that case. With the release of Hive 2 they patched the problems found but they should never have been there in the first place. Found in the same investigation, Google had a lesser fault with Nest which was sending the post code of the user unencrypted, which has since been patched. It is only when caught or there is a high profile breach that companies take the steps to fully secure their customers information \cite{which}. 
	
	Security on IoT devices is a new implementation of an old problem, developers and companies sometimes don't employ effective security on the private data they store, moving to smaller internet connected devices doesn't change this fundamental problem. There are studies about and examples of cryptographic systems for microcontrollers, it is not something that cannot be done. However, perhaps it is too difficult at present for smaller development teams or has too much of a foot print in terms of computational resources, power or time that it gets pushed to the side. Data security is a fundamentally important concept and one that is necessary for the implementation of all applications, especially those that involve user's private data otherwise there is the potential for loss of money, intellectual property, goods, reputation and health. These problems affect both companies and the consumer.
	
	This project will look into creating a solution for the transmission of secure, authenticated and encrypted data across an unsecure network that can be easily implemented, has a small code size and acceptable performance and security considering the code is run on a low powered device. As it is low powered the power consumption will be monitored as that would be a consideration if the device was to be powered by batteries. The example used to work with is the secure transmission of a user's private temperature data. If they have a system that monitors the temperature of rooms, that data can be used to figure out when they are likely to be home or not. So, using a base platform in the user network that has access to the temperature sensors throughout the house, it takes the sensor data, signs then authenticate encrypts it before sending it to a remote server. 
	
	
	
%% project planning and the tools i have used such as git, arduino IDE, java ee eclipse, ubuntu for random keys, routers, power module, logbook?

\chapter{Project Tools and Planning}
\label{planning}
helloaslkdhaskjdh klajdsh ak;sdjfh asdf
] asdlfj hsalkdfjh asd;flk as
df asldjfh laskdjfh s;dpolkfjh as
df[SAIHDFASLKDJFH APESGFKA;EODLKHaskjfhAS;KLJLHakjsdh ALKSJDH laksjdh LASKJDH laskjdh ALKSJDH alksjdhlsjH
%Chapter of Background
%Take everything from poster and then add a whole lot
%Explain AE,signing,cipers, types of attack that my app may have to defend against, keys, nonces, 

\chapter{Background}
\label{back}


This chapter will briefly explain the cryptologys used to make the application secure and what kind of attacks this can protect against. 

\section{Public Key}

\section{Signature}

\section{Authenticated Encryption} 
%Chapter of Design
%What my app has, (how it evolved?) in terms of physical structure and technologies involved

\chapter{Design}
\label{design}



\section{IoT Platform}

The basic concept of this platform is an Arduino Due that takes the current temperature of the room from a DS1820 temperature sensor. Then that data is signed and encrypted with TweetNaCL before being transmitted, using an Ethernet Shield, across to an SQL server. A web application takes the SQL data decrypts, checks the signature is valid then displays on a website. 

graphic here pls

Why was the Due chosen, 32 bit?

DS1820 is a lost cost temperature sensor that is very accurate, 12 bits of precision? and is also low power. It can scavenge power from the data with the arduino and thus does not need it's own power source. 


For the prototype, an Ethernet Shield was used as it is much cheaper than a WiFi shield but ultimately completes the same job. The shield is a simple way to connect arduinos to the internet. The shield used was the second revision, R2 and has a w500 ethernet controller. 


What are some of the options for the base station, Due/MSP430?  And for the internet connection   Ethernet?WiFi shield?

\section{Server Side}

For the prototype, an Apache server, SQL server and Tomcat server was set up using XAMPP on a desktop. A Java web app was created as that is the language the writer has the most experience in and there are Java implementations of the TweetNaCl library, among other variations, available. The SQL table is a simple table that holds a key, timestamp and the signed and encrypted temperature sensor. (example?). The web app upon being accessed decrypts and checks the signature of each entry, using the keys that it has stored, in the table before converting the raw hex temperature data into more readable integers and displaying in a simple HTML table that can be accessed by the user. When the Arduino has data to send it will make a POST request to a PHP file on the Apache server which takes the data given to it and places it in the SQL server. (security flaw!)

\subsection{XAMPP}

explain what XAMPP, Apache, SQL and Tomcat are. (maybe should be in background)
Xampp is a 

\subsection{Java web app}

The Java Web app details what the server is do when it gets various types of request, be it get or post...(more on requests?). In this type of application you can dynamically printout all the HTML that will be used to make up the page. The usual  HTML, head, body tags are printed at the top and the titles in the table are printed as well. (How it gets the keys?!). The web app uses JDBC to create a driver(idk man) to get the connection to the SQL database. Then using Java language it builds up a SQL query to take out all the values from the database and executes that. This puts all the table entries into a result set and the app cycles through that results set getting the relevant information out. The signed and encrypted hex is encoded as string and some leading zeros are lost in the conversion from byte array to string in the Arduino so these are added now before the string is converted back into a byte array. There is a try catch around the crypto\_box\_open and crypto\_sign\_open method so the server doesn't crash if one result set has been broken. Following this is the conversion from hex into integer for the user to read (how does it do it?) and finally the values are written to the browser along with the ending html tags.

\subsection{PHP}
PHP is a server-side scripting language 

There are two files in the server, connect.php and add.php. The Arduino makes a post request to the add.php which effectively just calls it and the first thing the add file does is call connect which has the server details and makes creates a connection. Following that there is a SQL query that inserts the values sent in the post request into the appropriate table entries then close the connection. 

\section{NaCl}

The keypair, crypto\_sign, crypto\_box and equivalent opens were used. These are simple to use, abstracted methods that make this library easy to use. For the encryption the method needs the message to be encrypted which needs to have the first 32 bytes be zero, an empty array that needs to be at least the size of the message with the leading zeros, the length of the message, the nonce, arduino public key and the servers private key. This will reveal the temperature data with the signature. To remove the signature, the crypto\_sign\_open method needs the server secret signature key and the signed cipher array.
%Chapter of Implementation
%What my app has, (how it evolved?) in terms of physical structure and technologies involved

\chapter{Implementation}
\label{imple}

\section{Overview}

The Arduino Due asks for the server's new public key and for a new nonce that it should use for the following temperature data transmission. It does this by sending a GET request to a JSP page on the server. The public key and nonce are newly generated by the server and the nonce is sent securely. The Due is connected to the DS18S20 temperature sensor and it receives the temperature data over the OneWire protocol. This data is signed using the Arduino's private signature key and then encrypted using the Arduino's private key, the Server's new public key and the new nonce. The encrypted data is sent as a POST request to the add.php file on the apache server which executes the file and the first thing add.php does is call connect.php which has the SQL database details and makes a connection to the database. Following that add.php builds up a SQL query that inserts the values sent in the post request into the appropriate table entries and then the connection is closed.
When a user want to view the files, they use a browser to send a GET request to the Java web app. The app builds up a SQL query to take out all the values from the database, decrypts them and verifies the signature before printing out in a table.
When it comes to public key transmission the Due sends a POST request to the Arduino Uno that is set up as a web server. The Uno recognises that it is receiving a POST request and stores the key. A visual representation of these steps can be seen in figure \ref{fig:messeq}.




\begin{figure}[H]
 % \centering
\begin {sequencediagram}

	\newthread[blue]{due}{: Arduino Due }
	\tikzstyle{inststyle}+=[bottom color=babyblueeyes]
	\newinst [1]{server}{: WebServer}
	\newinst {ds}{: DS18S20 Sensor }
	\newinst {uno}{: Arduino Uno }
	
	\begin {call}{due}{GET request}{server}{Key/Nonce}
	\end {call}
	%\begin {sdblock}{ Run Loop }{ The main loop }
	\begin {call}{due}{ GetTemp}{ds}{}
	%\begin {call}{ctr}{ ActAgent () }{sense}{}
	%\end {call}
	\end {call}
	\begin {call}{due}{ SendSecureTemp}{server}{}
	%\begin {messcall}{ps}{ PrePhysicsUpdate () }{sense}{state}
	%\end {messcall}
	%\begin {sdblock}{ Physics Loop }{}
	%\begin {callself}{ps}{ PhysicsUpdate () }{}
	%\end {callself}
	%\end {sdblock}
	\end {call}
	\begin {call}{due}{ SendPublicKey}{uno}{}
	
	\end {call}
	%\begin {call}{ss}{ EndCycle () }{ctr}{}
	%\begin {call}{ctr}{ SenseAgent () }{sense}{}
	%\end {call}
	%\end {call}
	%\end {sdblock}
\end {sequencediagram}
  \caption{System sequence diagram}
  \label{fig:messeq}
\end{figure}

\subsection{Temperature reading}

The DS18S20 temperature sensor is wired up with a 4.7$\Omega$ pull up resistor, between the orange wires, on the bread board shown in figure \ref{fig:tempcircuit}. A pull up resistor is a resistor between the sensor and the positive power supply so that the signal will be a valid logic level if external devices are disconnected or a high impedance is introduced. It is connected to digital pin 9, yellow wire, because it can't be on pins 10, 11, 12, 13 as they will be used by the Ethernet Shield. When looking at the temperature sensor the furtherest left pin is V$_{dd}$ and normally this would be connected to the Arduino's 3.5v or 5v output but the DS18S20 is in parasite power mode which scavenges power off the middle data line, DQ. When the DQ pin is high some of the charge is stored in a capacitor that will be used to power the device when the data is being read. In parasite power mode both the GND and V$_{dd}$ are connected together, by the bluewire, and then to ground. The circuit diagram shown was created with Fritzing \cite{fritz}.

\begin{figure}[H]
	\centering
	\includegraphics[width=.5\linewidth]{Figures/TempSensor_bb.pdf}
	\caption{DS18S20 in parasitic power mode connected to Arduino}
	\label{fig:tempcircuit}
\end{figure}

%can explain full scratchpad memory, CRC, or hex to int conversion
The DS18S20's scratchpad memory is divided up into 9 bytes as shown in figure \ref{fig:dsmem}. The first two bytes are the ones of most interest as they store the actual temperature value. Bytes 2 and 3 are the high and low trigger registers, these can be used to trigger some action when the temperature goes above or below a certain threshold. Bytes 4 and 5 are only for internal use and cannot be overwritten. Bytes 6 and 7 can be used to calculate the extended resolution. Byte 8 holds the cyclic redundancy check which is an error detecting technique. The bytes with asterisks are values which stored in non volatile EEPROM, the rest are stored on volatile SRAM. The shown table was taken from the DS18S20 datasheet \cite{dsdatasheet}.

\begin{figure}[H]
	\centering
	\includegraphics[width=.6\linewidth]{Figures/dsmem.png}
	\caption{DS18S20's memory organisation}
	\label{fig:dsmem}
\end{figure}

The library used to read the DS18S20 is OneWire, which is a proprietary library from Maxim that performs half-duplex bidirectional communications between a host/master controller and one or more slaves. As seen in the following code snippet, figure \ref{snip:tempcode}. The command \emph{ds.write(0x44, 1)} starts the internal A-D conversion operation. Once this process is finished the data is copied to the Scratchpad registers. A delay is needed to charge the capacitor and to ensure conversion is complete before reading the data. Which is done with \emph{ds.write(0xBE)} and then the data is read out using \emph{ds.read()} and put into an array.

\begin{figure}[H]
\begin{lstlisting}[style=Arduino]
  							OneWire  ds(9);
								...
  							ds.write(0x44, 1); 
  						 	delay(1000);

  							present = ds.reset();
  							ds.select(addr);    
 							ds.write(0xBE)

  							Serial.print("  Data = "); 
  							Serial.print(present, HEX);
 						 	Serial.print(" ");
  							for ( i = 0; i < 9; i++) {          
    								data[i] = ds.read();
    								Serial.print(data[i], HEX);
    								Serial.print(" ");
  							}

\end{lstlisting}
\caption{DS18S20 temperature sensor Arduino Code}
\label{snip:tempcode}
\end{figure}


\section{Security}

Before the data can be signed and authenticated encrypted the new nonce and server public key have to be requested. The Due sends, to the server, a GET request in a very similar manner to the POST request shown in figure \ref{snip:ethernet} and the server returns, byte by byte, the information. The POST request is sent to a JSP page called Secret Key. So called because it updates the Server's authenticated encryption key pair and not because it sends the private key anywhere. It's implementation is described in section \ref{pktransmit}. The Arduino takes the char arrays and converts them into two different byte arrays, one for the nonce and one for the public key. The public key is sent unencrypted but as a proof of concept the nonce is sent authenticated encrypted and signed. This is to show that with pre-installed keys it possible to update the keys and other data securely and then use the updated information to facilitate further information updates. Generally the nonce and public keys are sent unencrypted when it is a first transmission as there is no other way. Shown in figure \ref{snip:denonce} is the code to decrypt the nonce, if this is the first transmission then the nonce will have been encrypted with the pre-installed keys and nonce, else the server public key and nonce generated by the server for the previous secure temperature data transfer are used. The decision process is shown in figure \ref{snip:noncedec}. 

The way the Arduino Due knows if this is the first transmission is if the byte array \emph{serverpkold} is empty, as after the temperature data is encrypted with this new nonce and server public key they become the old nonce and server public key. After becoming the old nonce and server public key they are stored in different arrays so they can be used to decrypt the next nonce as the next nonce from the server will be encrypted using the private key that is mathetically linked to this ``old'' server public key and nonce.

  The keys used for the signature process and the set used for the authenticated encryption process are not the same. There are four key pairs used in this prototype. A key pair is a set of one public key and one private key. The Arduino Due has a signature key pair and an authenticated encryption key pair. The web server, also, has a signature key pair and an authenticated encryption key pair. The signature key pairs stay the same through out the application, so does the Arduino Due's authenticated encryption key pair. Of the keys, it is only the Server's authenticated encryption key pair that is updated.
The Due decrypts the nonce using the relevant keys and removes the 32 bytes of leading zeros that needed to be placed by the server for successful encryption before verifying the signature in line 19 of figure \ref{snip:denonce}. Nonces are 24 bytes in length and a signature is 64 bytes in length.

It is a bit unusual to update the server's public key so regularly. Normally there would be huge gaps of time between key updates because if the cryptographic library being used is good then it should take a long time to break. However sometimes keys become compromised and need to be changed so to highlight and demonstrate how keys could be updated, the server's keys were updated for each message.

\begin{figure}[H]
\centering
\begin{tikzpicture}[node distance = 4cm, auto]
    % Place nodes
    \node [decision] (decidekey) {Is this the first transmission?};
    \node[input, left of=decidekey](yes){Yes};
    \node[input, right of=decidekey](no){No};
    \node [blockf, below of=yes, node distance=3cm](seryes){Server: Encrypt Nonce with pre-installed keys};
    \node [blockf, below of=no, node distance=3cm](serno){Server: Encrypt Nonce with previous keys};
    \node[blockf, below of=seryes](dueyes){Due: Decrypt Nonce with pre-installed keys};
    \node[blockf, below of=serno](dueno){Due: Decrypt Nonce with previous keys};
    % \node [blockf] (init) {initialize model};
    %\node [cloud, left of=init] (expert) {expert};
    %\node [cloud, right of=init] (system) {system};
    %\node [blockf, below of=init] (identify) {identify candidate models};
    %\node [blockf, below of=identify] (evaluate) {evaluate candidate models};
    %\node [blockf, left of=evaluate, node distance=3cm] (update) {update model};
    
    %\node [blockf, below of=decide, node distance=3cm] (stop) {stop};
    % Draw edges
    \path [line] (decidekey) -| node [near end]{Yes} (seryes);
    \path [line] (decidekey) -| node [near end] {No}(serno);
    %\path [line] (yes) -- (seryes);
    \path [line] (no) -- (serno);
    \path [line] (seryes) -- (dueyes);
    \path [line] (serno) -- (dueno);
    %\path [line] (decide) -| node [near start] {yes} (update);
    %\path [line] (update) |- (identify);
    %\path [line] (decide) -- node {no}(stop);
    %\path [line,dashed] (expert) -- (init);
    %\path [line,dashed] (system) -- (init);
    %\path [line,dashed] (system) |- (evaluate);
\end{tikzpicture}
\caption{Nonce Decision Tree}
\label{snip:noncedec}
\end{figure}


\begin{figure}[H]
\begin{lstlisting}[style=Arduino]
#include <TweetNaCl2.h>
TweetNaCl2 tnacl;
int Suc_Decrypt;
int Suc_SignVerify;
int scnlenwz = crypto_box_NONCEBYTES + crypto_sign_BYTES + crypto_box_ZEROBYTES; 
byte scnoncenewtemp[scnlenwz]; //signed encrypted nonce with zeros
byte snoncewz[scnlenwz]; // signed nonce with zeros
int snlenwoz = crypto_box_NONCEBYTES + crypto_sign_BYTES;
byte snoncewoz[snlenwoz]; //signed nonce without zeros
int snlen = crypto_box_NONCEBYTES;
byte nonce[snlen];
	...
if(serverpkold[0]==NULL){
	Suc_Decrypt = tuit.crypto_box_open(snoncewz, scnoncenewtemp, scnlenwz, preinstallnonce, preinstallserverpk, arduinosk); 
}else{
	Suc_Decrypt = tuit.crypto_box_open(snoncewz, scnoncenewtemp, scnlenwz, nonceold, serverpkold, arduinosk);
}

Suc_SignVerify = tuit.crypto_sign_open(nonce,&nlen,snoncewoz,snlenwoz,serverpksign);

\end{lstlisting}
\caption{Decrypting the next nonce}
\label{snip:denonce}
\end{figure}

Now that the server public key and nonce have been found the secure temperature data transmission can take place. Much like the server, when the Arduino is encrypting a signature and message it must have padded out the data with 32 bytes of leading zeros specified in the NaCl website. Take special care that exactly 32 bytes are placed in front because if there isn't the correct number the encryption process will still complete without errors. However the decryption will fail and it isn't apparent that that error occurred when data was encrypted. The code used to sign and encrypt the data is shown in figure \ref{snip:nacl}.

It is a little unusual to employ both authenticated encryption and a signature process. This was done to show off more of the capabilities of the TweetNaCl library and to provide an extra layer of authenticity as an attacker would need another key pair to fully exploit the message.

\begin{figure}[H]
\begin{lstlisting}[style=Arduino]
#include <TweetNaCl2.h>
TweetNaCl2 tnacl;

byte arduinosk[crypto_box_SECRETKEYBYTES] = {...};
byte arduinosksign[crypto_sign_SECRETKEYBYTES] = {...};
byte serverpk[crypto_box_PUBLICKEYBYTES]  = {...};
byte nonce[crypto_box_NONCEBYTES] = {...};

int const messageLength = crypto_sign_BYTES + 9;
byte message[messageLength] = {...};
unsigned long long signedMessageLength=0;
byte signedCipher[signedMessageLength];
unsigned char signedMessage[messageLength+crypto_sign_BYTES];

tnacl.crypto_sign(signedMessage, &signedMessageLength, message, messageLength, arduinosksign);
tnacl.crypto_box(signedCipher, signedMessage, signedMessageLength, nonce, serverpk, arduinosk);
\end{lstlisting}
\caption{TweetNaCl Arduino Signature and Encryption Code}
\label{snip:nacl}
\end{figure}

The C TweetNaCl library has been converted into an Arduino library. It therefore needs an instance of TweetNaCl created which in the sketch is called \emph{tnacl}. With this instance the methods needed can be accessed. In this prototype some keys are preinstalled, the Arduino's authenticated encryption key pair, it's signature key pair, the server's signature key pair and the first nonce and first server authenticated encryption key pair that will only be used once to decrypt the first nonce. With each transmission the server's authenticated encryption key pair is updated and the Due will get a new nonce and server public key. 

Lines 4-13 set up the message byte arrays that are to be passed between the methods. \emph{message} is initialised as a byte array of size \emph{messageLength} which is \emph{crypto\_box\_ZEROBYTES}, 32 bytes, plus the length of the message. The first byte of the message contains the Arduino's unique identifier to prove the Arduino was the original sender and the second byte contains the server's unique identifier to prove that the message was meant for the server. \emph{signedMessage} and \emph{signedMessageLength} are passed in by reference so after \emph{crypto\_sign()} is complete the message with the signature and the size of that array will be in those variables, respectively. The resulting signed message needs to have 32 bytes of leading zeros added to it before encryption. The function \emph{crypto\_box} completes the authenticated encryption and places the cipher into \emph{signedCipher} which is passed in by reference. 

\clearpage

\section{Data transmission}
Once the data has been encrypted it is to be packaged up and sent across the network.

\begin{figure}[H]
\begin{lstlisting}[style=Arduino]
#include<Ethernet2.h> //Ethernet Shield R2 library
#include<SPI.h>

byte mac[] = {
0x90, 0xA2, 0xDA, 0x10, 0x2D, 0xD6 }; //MAC address of the Ethernet Shield
char server[] = "192.168.0.6"; //IP of apache web server
EthernetClient client;
IPAddress clientIP(192, 168, 0, 30);

 if(Ethernet.begin(mac)==0){
      Serial.println("Failed to assign IP");
      Ethernet.begin(mac, clientIP);
  }else{
     Serial.println("Assigned IP");
  }
if(client.connect(server,80)){
     String data = "temperatureHex=";
     int contentLength = data.length()+final.length();
     Serial.println("Connected");   
     client.println("POST /tempLog/add.php HTTP/1.1"); 
     client.println("HOST: 192.168.0.6"); 
     client.println("Content-Type: application/x-www-form-urlencoded");
     client.print("Content-Length: ");
     client.println(contentLength);
     client.println();
     client.print("temperatureHex=");
     client.print(final);
   }else{
       Serial.println("Failed to Connect");
   }
   client.stop();
\end{lstlisting}
\caption{Ethernet interfacing and transmission Code }
\label{snip:ethernet}
\end{figure}

 Just before the cipher is sent the byte array is converted into a String so that it can be passed around easily as one parameter. This is completed simply by cycling through the byte array and adding each entry together. Care has to be taken when there are hexadecimals that are 0x0F or lower. The leading zero is lost during the conversion to String which means that when the cipher reconverted into a byte array, it is incorrect and cannot be decrypted. This is solved by explicitly adding an extra ``'0'' String if the byte is less than or equal to 0x0F. The Arduino Due needs to know the MAC address of the Ethernet shield if it is to make contact with the server. A media access control address, MAC is a unique identifier assigned to network interfaces. It allows the router to know what device the data is being sent from and where to send the replies. It is not dynamic. The device now needs an IP address which is completed through the Dynamic Host Configuration Protocol, DHCP, by the router. The router running DHCP dynamically allocates network configuration parameters such as IP addresses to devices so that they automatically get one that isn't in use. This is completed with the line \emph{Ethernet.begin(mac)} on line 10 in figure \ref{snip:ethernet} which returns a 1 on successful IP allocation or 0 on failure. If it fails then a static IP can be assigned manually by \emph{Ethernet.Begin(mac, clientIP)}. A connection to the server is attempted using the IP and the port number, 80 in this case. It is worth mentioning here, if the server is running on a Windows OS then make an exception for the IP address of the Arduino Due otherwise Windows firewall will block it. If it successful the information is transmitted as a POST request, a POST request is a request method in the HTTP protocol. When a server receives a POST request it knows to take the data and complete an action with it. The request makes it known that it wants add.php to be executed upon receiving the data. %extra params, the content size and urlencoding
Once the data is sent the connection can be closed and other actions performed on the Arduino.
 
\section{Server Side}

For the prototype, an Apache and Tomcat server with SQL was set up using XAMPP on a desktop. The Arduino Due causes the add.php to be run and the first thing that add does is call connect.php that creates a connection to the relational database using SQL's IP address, username, password and the name of the database. If it can't connect it abandons the task and returns an error message. On successful connect it returns the connection variable.

\begin{figure}[H]
%\begin{lstlisting}[language=PHP]
\begin{lstlisting}[style=PHP]
<?php
   	include("connect.php");
 
   	$link=Connection();
	
	$temp=$_POST["temperatureHex"];
 
	$query = "INSERT INTO tempLog (tempHex) 
		VALUE ('".$temp."')"; 
 
   	mysql_query($query,$link);
   	mysql_close($link);
 
   	header("Location: index.php");
?>
\end{lstlisting}
\caption{Arduino to SQL interfacing code}
\label{snip:php}
\end{figure}

After connecting, control is handed back to the add file where the data to be stored is extracted out of the POST request and placed in a variable. Then an SQL query to insert the variable into the correct table is built up before being sent to the SQL server and the connection terminated. The SQL table contains an ID, the time at which the temperature data was received and the data and is created using the command shown in figure \ref{snip:sql}. Note between the database creation code and the PHP file the corresponding variables, tempLog, the table name and tempHex, the data.

\begin{figure}[H]
\begin{lstlisting}[style=SQL]
CREATE TABLE `iotplatform`.`tempLog` ( `id` INT(255) NOT NULL AUTO_INCREMENT , `timeStamp` TIMESTAMP on update CURRENT_TIMESTAMP NOT NULL DEFAULT CURRENT_TIMESTAMP , `tempHex` VARCHAR(100) NOT NULL , PRIMARY KEY (`id`)) ENGINE = InnoDB;
\end{lstlisting}
\caption{SQL database creation code}
\label{snip:sql}
\end{figure}
%InnoDB?


\subsection{Decryption and Temperature Data Display}
\label{decryptemp}

The data stored in the database is still encrypted, now a way is needed to decrypt it and display it to the user. A Java web app, using Java server pages, JSP, was created as there are Java implementations of the TweetNaCl library, among other variations, available\cite{ian}. Eclipse JEE Mars was used to create a dynamic web project which extends HttpServlet for the creation of dynamic web pages. In this you override at least one of the following methods, doGet, doPost, doPut and doDelete. Therefore depending on the type of HTTP request received a different action will occur. This application has the code in the doGet as the server will receive a get request when it is accessed by a user.

\begin{figure}[H]
\begin{lstlisting}[style=Java]
protected void doGet(HttpServletRequest request, HttpServletResponse response) throws ServletException, IOException {
response.setContentType("text/html");
PrintWriter printWriter  = response.getWriter();
printWriter.println("<html>");
}
\end{lstlisting}
\caption{Prepping the client printer in JSP}
\label{snip:clientprinterjsp}
\end{figure}

Using the code shown in figure \ref{snip:clientprinterjsp} the headers and tags you would need and expect in a HTML page can now be printed dynamically much like printing to a console. The keys are defined similarly to the code on the Arduino except the server has it's own secret key and the Arduino's public key.

\begin{figure}[H]
\begin{lstlisting}[style=Java]
	final String DB_URL="jdbc:mysql://localhost/iotplatform";
	String user = "root"; 
	String password = "";
	try
	        {
	          // Register JDBC driver
	          Class.forName("com.mysql.jdbc.Driver").newInstance();

	            // Open a connection
	            Connection conn = DriverManager.getConnection(DB_URL, user, password);

	            // Execute SQL query
	            Statement stmt = conn.createStatement();
	            String sql;
	            sql = "SELECT id, timeStamp, tempHex FROM tempLog";
	            ResultSet rs = stmt.executeQuery(sql);
	            // Extract data from result set
	            while(rs.next()){
	               //Retrieve by column name
	               int id  = rs.getInt("id");
	               String tempHex = rs.getString("tempHex");
	               Timestamp timeStamp = rs.getTimestamp("timeStamp");
	}
}
\end{lstlisting}
\caption{Accessing the SQL database in JSP}
\label{snip:jspcode1}
\end{figure}

In figure \ref{snip:jspcode1} is a section of code running on the Apache Tomcat server. The Java application was exported as a WAR file from Eclipse before being stored in the tomcat directory. The app is given the SQL details before entering a try catch that prints whatever the error message is to the client. Java database connectivity, JDBC, is an API for client access to a database. Before the Java app is exported as a WAR file it must have the JDBC bin jar in the lib folder in WEB-INF for the eclipse project otherwise it can't connect to the database\cite{jdbc}. The app gets a new instance of JDBC and then opens a connection to the server before building up a query to pull out the contents of the table. The variables are put into a result set which can be used to access each individual variable with the string name as a parameter. The encrypted message is still in it's String format and needs to be converted back into a byte array. This implementation used for String to byte array conversion was found externally\cite{byte2}. During the conversion to a String on the Arduino half of the leading zeros are lost but they are added again before the conversion to byte array. The Java implementation of TweetNaCl provides an extra layer of abstraction but with a slight modification to the Java file it can use the same method names as the C library.

\begin{figure}[H]
\begin{lstlisting}[style=Java]
TweetNaCl.crypto_box_open(signedMessage, cipher, messageLength, nonceCollection[sqlentry], arduinoPublicKey, keyCollection[sqlentry]);
byte[] javaPlainTextMessage = TweetNaCl.crypto_sign_open(signedCipherArray, ArduinoPublicSignatureKey);
\end{lstlisting}
\caption{Decryption of message and verification of signature in JSP}
\label{snip:decryptjsp}
\end{figure}

The security needs to be taken off in the reverse order as to how it was put on. The nonces and server secrets keys used have been stored in the order they were created. Which corresponds with the order of the data in the SQL database. As each entry from the table is removed the corresponding private key and nonce , from the 2D arrays, is used. \emph{TweetNaCl.crypto\_box\_open} passes the decrypted message out by reference but for \emph{TweetNaCl.crypto\_sign\_open} the message without the signature is returned. This is an example of the top layer of abstraction in Ian Preston's implementation of TweetNaCl. The  \emph{TweetNaCl.crypto\_box\_open} method is the exact same as the the C library but \emph{TweetNaCl.crypto\_sign\_open} is a slightly built upon method that has fewer parameters, completes more work behind the scenes and returns the byte message. The two unique identifiers can now be accessed and checked to show the Arduino sent the message and that it was meant for the Server. The variable \emph{javaPlainTextMessage} is the temperature in plain hexadecimal but it still needs conversion to integer so it can be read by the average user. The code is taken from the DS18S20 example from Arduino\cite{onewire}. To access the page in this project, the url \emph{192.168.0.6:8080/IoTPlatform/DecryptTemp} should be used. If the IP address isn't suffixed with the port number the browser will access the wrong part of the server.

\section{Encrypted Nonce transmission}
\label{nonce}
A new nonce needs to be created for each new communication. This is done using Java's psuedo random number generator from the class Random. This new random nonce is stored in an array in the web application so that the application can decrypted and display the data later. Then the nonce is signed with the server's signature, which doesn't change in this application. If this is the first transmission and therefore the first public key and nonce to be produced, the nonce with signature is encrypted using the pre-installed keys otherwise it will use the secret key and nonce created in the previous transmission. The encrypted nonce is printed out to the Due with two terminating characters either side, ``()''. This process is to show how information might be updated in the Arduino Due securely.

\section{Public Key Transmission}
\label{pktransmit}

For a secure connection public keys must first be sent. At the start of the Arduino due's code it sends a GET request, very similar to the POST request shown in figure \ref{snip:ethernet}. The request is sent to a Java page called Secret Key. This page is similar in set up to the temperature data display page. When accessed the page calls \emph{TweetNaCl.crypto\_box\_keypair()} to create the server's private key and corresponding public key. It stores the private key in an array and prints out the public key to the Due with two terminating characters either side. Along with the next nonce, this process is explain in section \ref{nonce}. During debugging the state of the web application can be printed out as it is being run. The code used to extract the public key is very similar to the code shown in figure \ref{snip:post} expect it watches for the character < to signify that the next word will be the key. Afterwards the key needs to be converted back into a byte array from a string before it can be used. The code used to convert into a byte array was adapted from external source\cite{String2}. The key is then used to encrypt the next temperature data transmission before being copied over into a second array so that it can be used to decrypt the nonce that will be sent encrypted from the server for the secure temperature data transmission after the next one. 

For the creation of other nodes on the network that are connected onto the Due, an Arduino Uno was set up as a host and example node. As two clients can't directly connect together with the Ethernet protocol and since the Arduino Due is a client to a web server, the Uno must be a server.
The Due sends a POST request to the Uno, very similar to the one it sends to the XAMPP web server, in figure \ref{snip:ethernet} however the key is under the content header. The method that the Due uses earlier isn't quite appropriate here as there is no PHP file on the Uno to run. Instead the key can be sent under an extra header in the POST, \emph{``Content: key''}.

\begin{figure}[H]
\begin{lstlisting}[style=Arduino]
String incomingWord = " ", key = "";
int saveNextWord = 0, takeData = 0;
 if (client.available()) {
        char c = client.read();
        Serial.write(c);
        if(c == ' '){
          key = incomingWord;
          incomingWord=" ";
        }else{
          incomingWord = incomingWord + c;
        }
        if((incomingWord=="Content") && (takeData)){
           saveNextWord = 1; 
        }
        if(incomingWord=="POST"){
          takeData=1;
        }
}
\end{lstlisting}
\caption{Handling POST requests in Arduino}
\label{snip:post}
\end{figure}

When setting up the Uno server, explicitly define what gateway and subnet the router being used has as the defaults \emph{Ethernet.begin(mac)} uses aren't always correct. The server sits open on a certain port, 8081 in this case, and waits for clients to make a connection. Once they have, the server sends an acknowledgement that it has received a connection and starts reading in the request. As shown as in figure \ref{snip:post} the Uno watches out for the word POST so that it knows to store the content and, of course, for the content so it knows what to store. The request from the client is read out a byte at a time so it is necessary for those bytes to be made into full words so specific keywords can be searched for. If the byte is not a space then it is part of a word so it is added to the subsequent bytes until another space is reached then it is considered a word and compared against. Once the key is taken it is in String format and needs to be turn back into hex and stored as the key to be used in further encryptions.


\section{TweetNaCl Library}

The TweetNaCl library as it stands in it's original form is not compatible with Arduinos. The C library compiles without errors but the compiler warns that the TweetNaCl method names are undefined and as a result do not perform their tasks. When the methods are accessed they simply return random numbers. To get the libraries to work they were converted into C++ syntax. With a header file that has the main methods used in the project and the \#defines and a cpp file with the TweetNaCl code. This is added as a library to Arduino IDE and in the code an instance of the class is created and methods are accessed with the dot operator. In this application not every function in the TweetNaCl library will be used so any methods that weren't going to be utilised were not copied over into the converted library.



 
%Chapter of secureness
%How safe is my app, what attacks can it stand up against and why

\chapter{Strength Of Security}
\label{stre}

%Has this become the results section??
%In the result, you simply display results, don't discuss, that's for critical evaluation

\section{Sign then Encrypt}

Generally encryption stops unauthorised parties from accessing information, assuming authorised parties are the only party with access to a secret key, prove message integrity. Signatures prove that the message came from a known sender and the sender cannot deny sending the message. These are two distinct operations and the order of these operations matters. If a user was to encrypt a message then sign and send it, as the signature is added to the message, it can be stripped off by an attacker and replaced. If the attacker managed to intercept the transmission of public keys in the first place then they can now impersonated the real recipient. Signing after encryption defeats the purpose of signing, which is to prove that the owner of the signature saw the data and cannot disprove that if their signature is attached to it. If an encrypted message is signed then it is entirely possible that the owner of the signature was not aware what was in the message. In this application the message is signed first so the hash is taken of the plaintext message and it can be proved that the signer was aware as to the contents of the message and then it is encrypted\cite{signencrypt}.


\section{Storing data in plaintext}

Storing vital data such as passwords and usernames in plain text in a database is generally considered quite a bad idea. This provides one point of failure that if broken means that every password stored is now useless and the accounts are wide open. In some circumstances that initial break might not seem so severe, say if that is for a forum site where the worst action that could be taken was writing an inappropriate message but users sometimes use the same password for many different sites. Although in this application the information being stored in temperature data and not passwords but the security implications are still there. This application would have that single point of failure that is so deplorable. It is necessary to provide as many layers of security as possible rather than have one hard layer. That being the security of the database. As such this application stores all the temperature data as it arrives, in it's encrypted form. This way the keys will need to be known by the attacker in order to gain the data.
%should be in hash or something better as we'd need to store all the keys and work out what key to use on what data if we ever wanted access.

\section{Public Key Transmission}

If the keys are preinstalled they can be used to sign and encrypt new public keys before they are sent. So that the receiver can use the presinstalled keys to prove integrity and authenticity of the new keys. If there are no preinstalled keys then there is no way to prove that the public key being received is indeed the public key sent. At least for a computer. The most secure and safe method to make sure the keys are the same is to have a human user manually inspected the key before sending and after sending and carefully ascertain that the one sent is correct. If there has been a man in the middle, MITM, and either the key has been altered or replaced entirely then the human user can see the difference. 

\section{Bit Flipping}

Bit flipping is the process of changing parts of the encrypted message so that it says something else. This is especially potent when the format of the message is know. As the format of the message is know to us, a test can be set that if it succeeds would change a number. In this project a test was set up to demonstrate the cryptography caused detected that the message had been altered in transit. After a signed message was encrypted it was altered with random hexadecimal numbers and it each time the decryption process would fail. The encryption process adds a hash of the message and the decryption process would find a different hash and would fail each time.

\section{Timing Attacks}

TweetNaCl is an example of constant time software, which means that the time of execution does not depend on secret data and is therefore not vulnerable to timing attacks. To have this quality TweetNaCl avoids all loads from address and all branch conditions that depend on secret data and it is thus inherently protected against cache timing attacks. 

\section{Replay Attacks}

This attack is the act of resending of captured valid messages to repeat a valid action, like sending money or authentication. The potential for serious damage by a replay attack here is limited as this application. If an attacker resends some data to the server, say a malicious set of temperature data, when the server would go to decrypt that message it would try to use the incorrect nonce and public key and the message would be ignore, thus protecting the application from replay attacks.

\section{Length of cipher text}

Is it possible to gain information about the message from looking at characteristics of the cipher text? In bad cipher text it can be possible to find out the types of message that are sent as the resulting ciphertext may have change in a certain, predictable. For example if the message is a large number in contrast to a low number, there may be noticeable differences. Unfortunately a vulnerability exists in the cryptography chosen but it is a common weakness and one that does not have much consequence. That is the length of the cipher text corresponds with the length of the message. This is offset somewhat by the additition of the signature so the cipher text is much larger than one would expect if it was know that 9 bytes of hexadecimal were sent. To fully get around this the message is padded out with zeros so that the whole message is always forty one bytes in length. Which does mean the message has a limit of forty one bytes unless the source code is altered. To test how the cipher text changes when sending different messages four encryptions took place. The messages were YES, NO, 1 and 9999. 

\begin{table}[H]
	\centering
	\begin{tabular}{ | l | p{7cm} | }
	\hline
	Plaintext Message & Encrypted Message \\ \hline
	YES & 0xAE, 0x43, 0xCD, 0xA5, 0x8E, 0x54, 0xE9,  0x30, 0x59, 0xB1, 0xD5, 0xA5, 0xBF, 0x24, 0x9D, 0xEE, 0x69, 0xDB, 0x37  \\ \hline
	NO &  0x2E, 0xBC, 0xCC, 0x6B, 0x9E, 0xDB, 0x29, 0x64, 0xF6, 0x26, 0x49, 0xEF, 0xE9, 0xFA, 0xBD, 0x9C, 0x7E, 0xD1 \\ \hline
	\end{tabular}
	\caption{Resulting ciphers for encrypting words}
	\label{tab:yesno}
\end{table}

From theses message it is possible to tell the number of bytes sent if you know that the encryption method adds 16 extra bytes but other than that this test shows there isn't a discernible difference between encrypting YES and NO.

\begin{table}[H]
	\centering
	\begin{tabular}{ | l | p{7cm} | }
	\hline
	Plaintext Message & Encrypted Message \\ \hline
	00 & 0x5B, 0x74, 0x76, 0x4C, 0xF4, 0x19, 0x65, 0x37, 0xC4, 0x53, 0xAF, 0xB0, 0xCE, 0x18, 0x1, 0x1, 0x30, 0x9E \\ \hline
	1 & 0x49, 0x1B, 0x1E, 0x52, 0x10, 0x38, 0xD7, 0x38, 0x30, 0x65, 0x71, 0xBC, 0xEE, 0x65, 0x3D, 0x6, 0x31, 0x9E\\ \hline
	99999 & 0xC, 0xC, 0x29, 0xE2, 0x4E, 0x81, 0x67, 0x5, 0x78, 0x49, 0x8C, 0xA1, 0x4F, 0x69, 0x8, 0xBB, 0x9, 0xA7, 0x5D, 0x63, 0x4D \\ \hline
	\end{tabular}
	\caption{Resulting ciphers for encrypting numbers}
	\label{tab:9999}
\end{table}

Again, the bytes return from encrypting the test case of numbers don't indicate that the integer contained in the message is increasing but it does show the length of the number.



%getting the encryption and decryption to have the same nonce

%Chapter of Results
%Lets see it in action
\chapter{Results}
\label{res}

link to objectives

\section{Recording of temperature}

Using the thing it can record temp, sign, encrypt to ward off the attacks mention in the background. The temperat

\section{Sending Data}

The device successfully transmits data in the form of a POST request across to a web server.
\section{Power Consumption!}

\section{Web App}

Users can access the temperature table anywhere 

\section{Secure transmission of Public keys}

Two arduino Dues in the same network can send their public keys secrurely and a person has checked that

show lots of pictures?

\section{Machine to Machine}

Two arduinos sent up in a client host relationship were able to pass data


%Chapter of Critical Evaluation
%What is maybe not so great about it
\chapter{Critical Evaluation}
\label{crit}




\section{SubSection}

\section{Other SubSection}
%Chapter of Conclusion

% Review and summarise points
% What do I want the reader to come away with - Security is important, can be easy with a small amount of fuss, can be done on IoT, why arduinos are good and 
% how they will help encryption on small systems as they are affordable but powerful, how important IoT will be 

% design/feasability report
% Major conclusion
% Reference to aims and objectives then advantages of findings
% support for choice
% authors evaluation

% feasability/recommendation report
% background to problem
% research problem
% major conclusion
% support of conclusion
\chapter{Conclusion}
\label{conc}

This report has shown that is it possible to provide low power, not overly complicated encryption and authentication on a micrcontroller based IoT system. TweetNaCl is an excellent library that can be added to any project easily, has tiny code size but still provides acceptably strong encryption at a quick speed. It is a full library and can produce asymmetric and symmetric encryptions plus a SHA-512 hashing function and a string comparison function which any system can implement to cover it's encryption needs.

In any computer application which handles information that is private and/or could be used to do damage such as a persons credit card details meaning that person loses money or the lose of private user data that causes the loss of trust and damage to the reputation of a company, needs good encryption. As more and more devices are connected to the internet such, as smartphones, smart televisions, smart domestic appliances and anything else an IoT engineer thinks might benefit from an internet connection. As the entire planet shifts more and more into the digital world, this need grows ever larger. When data, such as credit card details, are lost this may be a huge problem but it is not fatal. However with the advent of heavy machinery such as cars becoming part of the internet of things, and other examples no doubt soon to follow, security should be at the forefront of everyone's minds. This project has aimed to show that acceptable encryption can be implemented easily, even on low power systems. In the prototype the users private temperature data is safe from attackers. 

Arduino too, was a good choice as Arduinos have been use for fast prototyping and for teaching people who might not have a background in electronics or computing how to create projects, such as IoT projects. There a multitude of internet connected projects for the Arduino, many connect to benign objects such as the LEDs on the board but they can be connected to many more things, some which can be troublesome if compromised. Security is generally an after though, for experienced and novice developers alike. Even massive companies with millions pounds worth of revenue make security blunders or just don't bothered. It is hoped that this project will add to the list of secure, internet connected Arduino projects available so that one of the first things of everyone's mind when starting a project is security. The powerful but lost cost Arduino devices prove that it is possible to have good quality security in any IoT system and more security options will pop up as the price to performance ratio continues on it's upward trajectory. Which is very beneficial to the IoT industry as a whole.

Security measures will never be fully secure. Given infinite computer resources and infinite time any security algorithm can be hacked and there might bugs, backdoors, employees willing to leak secret keys. However if the algorithms make it unfeasible to attack or simply not worth the attackers effort required then that is as good as fully secure. At the moment there are so many devices that simply have no protection that there is an open source project google powered search engine that shows all the devices that are vulnerable. These devices lack even rudimentary security features and even a casual attacker with not too much knowledge or experience could gain access and control. By having even the minimal amount of protection, obviously the more protection the better, you can dissuade would be attackers because they have plenty of easier targets. Or make it so that the reward of breaking into your system simply is not worth the effort.

The TweetNaCl library satisfies all the requirements of a good microcontroller cryptolibrary. It is open source and the developers encourage it's use where ever possible. It was used with success in this project as the users private data could not be read in transit nor altered in transit and the receiver could prove that a certain person sent the message. 
 
%\include{ further work }

%----------------------------------------------------------------------------------------
%	THESIS CONTENT - APPENDICES
%----------------------------------------------------------------------------------------

%\appendix % Cue to tell LaTeX that the following "chapters" are Appendices

% Include the appendices of the thesis as separate files from the Appendices folder
% Uncomment the lines as you write the Appendices

%% Appendix A

\chapter{Appendix: Source Code} % Main appendix title

\label{AppendixA} % For referencing this appendix elsewhere, use \ref{AppendixA}

\section{Server Side Code}

\subsection{Java Server Pages}

\subsubsection{DecryptTemp.java}

\lstinputlisting[language=Java]{Code/DecryptTemp.java}

\subsubsection{SecretKey.java}

\lstinputlisting[language=Java]{Code/SecretKey.java}

\subsection{PHP}

\subsubsection{connect.php}

\lstinputlisting[language=PHP]{Code/connect.php}


\subsubsection{add.php}

\lstinputlisting[language=PHP]{Code/add.php}


\section{Microcontroller}

\subsection{Arduino Due}

\begin{figure}[H]
\begin{lstlisting}[style=Arduino]

\end{lstlisting}
\caption{Handling POST requests in Arduino}
\label{snip:post}
\end{figure}


\subsection{Arduino Uno}

\begin{figure}[H]
\begin{lstlisting}[style=Arduino]

\end{lstlisting}
\caption{Handling POST requests in Arduino}
\label{snip:post}
\end{figure}

%\include{Appendices/AppendixB}
%\include{Appendices/AppendixC}

%----------------------------------------------------------------------------------------
%	BIBLIOGRAPHY
%----------------------------------------------------------------------------------------
%\bibliographystyle{plain}

%need to compile bib.bib
%then run bibtex main.aux
\printbibliography

%\begin{thebibliography}{99}
%\bibitem{lamport}
 % Leslie Lamport,
  %\emph{\LaTeX: a document preparation system},
  %Addison Wesley, Massachusetts,
  %2nd edition,
  %1994.
%\end{thebibliography}


%----------------------------------------------------------------------------------------
%	USEFUL COMMANDS I WILL FORGET
%----------------------------------------------------------------------------------------
% \emph{}   make the text italic
% {\tt ``'code'} make the text look like code
%----------------------------------------------------------------------------------------


\end{document}  
