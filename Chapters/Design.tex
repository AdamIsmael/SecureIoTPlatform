%Chapter of Design

\chapter{Design}
\label{design}

\section{Aims}

The basic aims are to design and implement a basic protocol for the transmission of DS18S20 temperature sensor data to a web server from an Arduino Due. This protocol will have security such that the messages are protected from modification, spoofing and can't be read in transit. There is to be a server application that allows the user to view the data being sent to the server.
Following on, the needs of machine to machine communication will be considered and the protocol will be expanded to include direct communication and transmission of keys to an Arduino Uno. Finally, as this will be on a low powered device, the power usage is to be measured to establish performance and power overhead of security. 

%``prevents rogue devices from transmitting data to others''?!?!

% The block diagram code is probably more verbose than necessary
\begin{figure}[H]
\centering
\begin{tikzpicture}[auto, node distance=2cm,>=latex']
    % We start by placing the blocks
    \node[block, name=web] {Web Server};
    \node[block, below of=web] (due) {Arduino Due};
    \node[block, right of=due,  node distance=4cm] (uno) {Arduino Uno};
    %\node[block, below of=due] (uno1) {Arduino Uno};
    \node[block, left of=due, node distance=3.5cm](ds) {DS18S20};
    
    \draw [->] (due) -- node[name=key] {$key$} (uno);
    \draw [draw,->] (ds) -- node {$data$} (due);
    %\draw [->] (due) -- node[name=key] {$key$} (uno1);
    \draw[->]([xshift=.3cm] web.south) -- node[name=key] {$key$}([xshift=.3cm] due.north);
    \draw[->]([xshift=-.3cm] due.north) -- node[name=key] {\emph{encrypted data}}([xshift=-.3cm] web.south);
     
\end{tikzpicture}
\caption{Node Diagram}
\label{dia:node}
\end{figure}


\section{Microcontroller}

The first step in this application is the acquisition of data. For example data, a variable that is monitored for important applications in the real world was chosen, this being temperature. There are a range of temperature sensors out there but one that is compact, cheap, freely available, has a large temperature range, high precision and can derive power directly from the data line, so called parasite power, is Maxim's DS18S20. It returns a 9-bit byte array so is ready to encrypt straight away. The image shown is taken from fluuuz.de \cite{dsimage}.

\begin{figure}[H]
	\centering
	\includegraphics[width=.4\linewidth]{Figures/ds.jpg}
	\caption{DS18S20 Temperature Sensor}
	\label{fig:ds}
\end{figure}

The Arduino Due is a large Arduino, it is the first one with a 32-bit ARM core micrcontroller, has 54 digital I/O pins, 512KB flash memory, 96KB SRAM and a 84MHz clock speed. A lot of microcontrollers are 8-bit which means that they are limited in their cryptographic options. With a 32-bit architecture and a large amount of flash memory it is possible to have a light weight cryptographic library for our application but the Due is still low cost enough that it can be used for an IoT sensor application. Because it is an Arduino it benefits from the large community, wide range of compatible components and large set of open source libraries. The image shown is taken arduino.cc \cite{dueimage}.

\begin{figure}[H]
	\centering
	\includegraphics[width=.4\linewidth]{Figures/due.jpg}
	\caption{Arduino Due Microcontroller}
	\label{fig:due}
\end{figure}

Once the temperature data is on the microcontroller there needs to be a way to transmit the data. As the data is being sent over a network the natural choice is an Ethernet or WiFi shield. The Ethernet Shield was chosen as it is much cheaper than the WiFi Shield but completes the same job. The first revision of the shield is no longer made so the second revision, R2, will be utilised. It allows for easy connection of the Arduino to the Internet, it uses the Wiznet W5500 Ethernet chip, supports up to 8 different socket connections at a speed of 10/100Mb and has a MicroSD card slot to store network settings. It needs an different library than the first revision\cite{eth2}. Fortunately the two libraries share the same API so software that was built for the first revision is compatible with the second. Simply swap put the older shield and library and replace with the newer one. The image shown is taken from arduino.cc \cite{ethimage}.

\begin{figure}[H]
	\centering
	\includegraphics[width=.4\linewidth]{Figures/ethernet.jpg}
	\caption{Arduino Ethernet Shield}
	\label{fig:eth}
\end{figure}

\section{Security}

There are a lot of security options for desktop programmes and communications like AES, WEP and SSL. In this more ubiquitous field the implementations have been around for a while but for microcontrollers only in recent times have they become powerful enough at a cheap enough price and therefore popular enough for developers to write or adapt cryptographic libraries suited for microcontroller communications. The library in this application had to have acceptable security, really as close as possible to encryption systems in more powerful computers but still have acceptable performance, have a small enough code size to be stored on the microcontroller and be easy to use. Of course, it needs to prevent attackers from reading or altering the data and proving who sent it. Some microcontroller applications have extra chips that are solely for encryption but that comes at extra cost.

%Are 4-8 bit crypto libraries just not as good as 32 bit?

TweetNaCl was chosen for this project. It is a public-domain, auditable and tiny implementation of NaCl, a high-speed cryptographic library. NaCl aims for absolute performance with good security at the expense of portability whereas the aim of TweetNaCl is portability, small code size and auditability. The developers claim it is first auditable high-security cryptographic library. It is a recently created library, released in 2014 and it provides public-key cryptography, secret-key cryptography, hashing and string comparison. TweetNaCl is a full cryptography library but has only two files, needs little to no memory to be stored, has similar performance to NaCl and is easy to use. It has a set of high level methods that require the necessary variables and returns the encrypted or signed message. It doesn't provide many options, you call the method with the message and keys and the software does the rest. This reduction in customisability reduces the likely hood of a mistake by a developer using the library.


\section{Server}

\subsection{Web Server}

The web server is needed to be able to complete the relatively difficult jobs of decryption, checking signature and integrity, pulling data from the SQL database and displaying the data. For this reason and the fact that there is a Java implementation of TweetNaCl from a developer called Ian Preston, a Java Server Pages application was set up running on Apache Tomcat. JSP is a tool to dynamically create web pages using Java code. JSP could be joined with the Java implementation of the C library, TweetNaCl, to retrieve the encrypted values from a database, decrypt and display them. For the database structured query language, SQL, was chosen. When the data is being sent to the web server it needs to be sent in a particular way. Data is passed into and taken from servers is using POST and GET requests. And it is possible to send a GET or POST request to a file and that file then executes. When the Arduino is sending data to the database it was decided the script it sent a POST request to would be a PHP file as taking the encrypted and signed data and storing it could be handled by a simple PHP script.

\subsection{Arduino Server}

If it is desired that the temperature be taken from more than one location then there will need to be multiple nodes on the network. Or if there needs to be extra microcontrollers on the network for some other task. One such task is the distribution of public keys, if the server is to update it's public key, say as part of a regular update schedule to provide good security or if a key is known to be comprised then it will need a way to pass on the key. The Due can request a new key or the server can indicate to the Due that a new key has been made available and the Due will make a request for that key then make a connection to the other Arduinos in the network and send them that key. An extra Arduino Uno was chosen as an extra node for testing. The network will be built using Ethernet and be an extension of the Arduino Due to web server relationship. As the Arduino Due is a client to the web server the other nodes will need to be hosts if information is to be passed between them. The setup described in figure \ref{dia:node} is desired. 