%Chapter of Results
%Lets see it in action
\chapter{Results}
\label{res}

link to objectives

\section{Recording of temperature}

Using the thing it can record temp, sign, encrypt to ward off the attacks mention in the background. The temper

\section{Sending Data}

The device successfully transmits data in the form of a POST request across to a web server.

\section{Power Consumption!}

is flashed the normal way
take temp reading every couple of seconds in parasite mode, 
4.96V, 0.123A, 613mW

flashed through device with empty code
power reading stay same

flashed normally with empty code
same

flashed normal and not
take temp reading every couple of seconds in parasite mode
same again

When reading the data from the temperature and encrypting said data
the current increases from .123A  .17 to .20 and
the wattage increases from 613mW to 970 for signing to 1000 for encryption

During the dhcp and connecting to server .199A and 990mW

During the busy wait at the end of loop delay() the current is .198A and 980mW
a tiny drop in current

\section{Web App}

Users can access the temperature table anywhere 

\section{Secure transmission of Public keys}

Two arduino Dues in the same network can send their public keys securely and a person has checked that

show lots of pictures?

\section{Machine to Machine}

Two arduinos sent up in a client host relationship were able to pass data

http://blog.skylable.com/2014/05/tweetnacl-carrybit-bug/