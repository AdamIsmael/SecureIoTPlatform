%Chapter of Results
%Lets see it in action
\chapter{Results}
\label{res}

link to objectives

\section{Recording of temperature}

Using the thing it can record temp, sign, encrypt to ward off the attacks mention in the background. The temper

\section{Sending Data}

The device successfully transmits data in the form of a POST request across to a web server.

\section{Power Consumption!}

So when the ethernet shield but no ethernet cable is plugged in, the platform starts with 4.96V, 0.123A, 613mW and stays that way when reading from the temp sensor as the sensor is in parasite moew, even if it wasn't it would not be noticeable compare to the idle power needs of the due. When the signing was ocurring it jumpred to .17A/970mW and for encrypt it was .20A/1000mW. Then during dchp when ethernet shield was in but not cable .199A and990W. With a tiny drop when it failed to connect and was in the busy wait at .198A and 980mW.

However with the ethernet cable plugged the power consumption shot up to 1.30w/0.265a and remained at that level even during the encryption process. Evidently the shield requires the lions share of the power and the cyrptograpphic library is good as to be unnoticable.


is flashed the normal way
take temp reading every couple of seconds in parasite mode, 
4.96V, 0.123A, 613mW

flashed through device with empty code
power reading stay same

flashed normally with empty code
same

flashed normal and not
take temp reading every couple of seconds in parasite mode
same again

When reading the data from the temperature and encrypting said data
the current increases from .123A  .17 to .20 and
					resting sign and encrypt
the wattage increases from 613mW to 970 for signing to 1000 for encryption

				and failing to connect		
				
During the dhcp and connecting to server .199A and 990mW with no ethernet plugged in

During the busy wait at the end of loop delay() the current is .198A and 980mW
a tiny drop in current


with ethernet plugged in, sending encrypted data
1.30w /0.265a

tested on data of 19 length and 209, made no difference

with ethernet plugged in but no encryption
1.30w /0.265a

\section{Web App}

Users can access the temperature table anywhere 

\section{Secure transmission of Public keys}

Two arduino Dues in the same network can send their public keys securely and a person has checked that

show lots of pictures?

\section{Machine to Machine}

Two arduinos sent up in a client host relationship were able to pass data

http://blog.skylable.com/2014/05/tweetnacl-carrybit-bug/