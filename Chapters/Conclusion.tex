%Chapter of Conclusion

% Review and summarise points
% What do I want the reader to come away with - Security is important, can be easy with a small amount of fuss, can be done on IoT, why arduinos are good and 
% how they will help encryption on small systems as they are affordable but powerful, how important IoT will be 

% design/feasability report
% Major conclusion
% Reference to aims and objectives then advantages of findings
% support for choice
% authors evaluation

% feasability/recommendation report
% background to problem
% research problem
% major conclusion
% support of conclusion
\chapter{Conclusion}
\label{conc}

This report has shown that is it possible to provide low power, not overly complicated encryption and authentication on a microcontroller based IoT system. TweetNaCl is an excellent library that can be added to any project easily, has tiny code size but still provides acceptably strong encryption at a quick speed. It is a full library and can produce asymmetric and symmetric encryptions plus a SHA-512 hashing function and a string comparison function which any system can implement to cover it's encryption needs.

Any computer application which handles information that is private, that could be used to do damage such as a persons credit card details meaning that person loses money needs good encryption. As more and more devices are connected to the internet such as smartphones, smart televisions, smart domestic appliances and anything else an IoT engineer thinks might benefit from an internet connection. As the entire planet shifts more and more into the digital world, this need grows ever larger. When data, such as credit card details, are lost this may be a huge problem but it is not fatal. However with the advent of heavy machinery such as cars becoming part of the internet of things, and other examples no doubt soon to follow, security should be at the forefront of everyone's minds. This project has aimed to show that acceptable encryption can be implemented easily, even on low power systems. In this prototype, the users private temperature data is safe from attackers. 

The Arduino was a good choice as Arduinos have been use for fast prototyping and for teaching people who might not have a background in electronics or computing how to create projects, such as IoT projects. There are a multitude of internet connected projects for the Arduino, many connect to benign objects such as the LEDs on the board but they can be connected to many more things, some which can be troublesome if compromised. Security is generally an afterthought, for experienced and novice developers alike. Even massive companies with millions pounds worth of revenue make security blunders or plain just don't bother. It is hoped that this project will add to the list of secure, internet connected Arduino projects available so that one of the first things of everyone's mind when starting a project is security. The powerful but lost cost Arduino devices prove that it is possible to have good quality security in any IoT system and more security options will pop up as the price to performance ratio continues to improve. Which is very beneficial to the IoT industry as a whole.

Security measures will never be fully secure. Given vast amounts of computer resources and vast amounts of time any security algorithm can be hacked and there might bugs, backdoors or employees willing to leak secret keys. However if the algorithms make it unfeasible to attack or simply not worth the attacker's effort then that is secure. At the moment there are so many devices that simply have no protection that there is an open source google powered search engine project that shows all the devices that are connected to the internet but vulnerable\cite{censys}. These devices lack even rudimentary security features and even a casual attacker with not too much knowledge or experience could gain access and control. By having even the minimal amount of protection, obviously the more protection the better, you can dissuade would be attackers because they have plenty of easier targets.

The TweetNaCl library satisfies all the requirements of a good microcontroller cryptographic library. It is open source and the developers encourage it's use where ever possible. It was used with success in this project as the users private data could not be read in transit nor altered in transit and the receiver could prove that a certain person sent the message. 
