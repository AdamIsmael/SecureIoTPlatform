%Chapter of Background
%Take everything from poster and then add a whole lot
%Explain AE,signing,cipers, types of attack that my app may have to defend against, keys, nonces, 

\chapter{Background}
\label{back}


\section{Cryptography}

Cryptography is the practise and study of techniques for secure communication in the presence of attackers. To do so, one can use encryption where by messages are encoded in such a way that only authorised parties or at least parties in possession of the keys can view them. There are two main ways of encryption Symmetric Key encryption and Public Key encryption. In Symmetric Key encryption both parties have the same key the which can encrypt and decrypt messages that are sent between them. The problem is that if Bob wants to send an encrypted message to Alice, he must get the secret key to her. Currently the most secure way for the transmission of secret keys is to hand them over in person, in private. To prove authenticity and integrity of the message, one can use HMAC

\section{Types of attacks}
play back/replay attacks
man in the middle attacks

stream cipher attacks, if the same key is used

\subsection{TweetNaCl}

NaCl or ``Salt`` is a simple to use high-speed library for authenticated encryption. it provides both Asymmetric and Symmetric encryption, It provides authentication and message integrity with SHA-512. The authors are Daniel J. Bernstein, Tanja Lange and Peter Schwabe but at points it relys on third part implementations for parts. The API is simple, having only a handful of methods but uses high speed, high security primitives.
/ref{https://labs.opendns.com/2013/03/06/announcing-sodium-a-new-cryptographic-library/}

Unfortunately the library wouldn't work completely on a Arduino, one problem is that there is no /dev/random and therefore no randombytes() which means that it can't create keypairs in the usual way.
Also, as mentioned later Arduino can't use the C library as is, it needs to be converted into C++. 
Another problem is that the library is relatively quite large, the Arduino Due has at it's disposal 512KB flash memory and the full library is 3MB. Fortunately the same creators along with Bernard van Gastel, Wesley Janssen and Sjaak Smetsers made TweetNaCl. Which is a tiny implementation of NaCl, still providing speed and security but with a significantly smaller code size 40KB. It retains the same protections against (from tweetNack-201409..) timing attacks, cache-timing attacks, has to branches depending on secret data and no array indices depending on secret data. In addition it is thread-safe and has no dynamic memory allocation. It is portable and easy to integrate, the library is easily added as it consists of two files, there is no complicated configuration to be set up or any dependencies on external libraries. Because of this compactness it is easier to read and understand it's operation. Although not as fast as NaCl it is still fast enough for most applications. ``Most applications can tolerate the 4.2 million cycles that OpenSSL uses
on an Ivy Bridge CPU for RSA-2048 decryption, for example, so they can certainly tolerate
the 2.5 million cycles that TweetNaCl uses for higher-security decryption (Curve25519).''
TweetNaCl is still small after compilation at 11KB thus avoiding instruction cache misses. 


\subsection{Public Key}

To get round the problem of securing sending secret keys, one can use Asymmetric Key encryption has a secret key and a public key, the public key is generated out of the secret key and are therefore mathematically linked but it is computationally infeasible to calculate the secret key from the public key. This public key can be given out freely and is not a secret. So if Bob sends a message to Alice he encrypts the message with her private key and she can decrypt it with her secret key.


\subsection{Signature}

The method used in this project to prove authenticity and integrity is a digital signature. 



\section{Technologies used}

php/java/C++/SQL


All Arduino code was created using the Arduino IDE and the server Java was created using Eclipse Jee Mars.