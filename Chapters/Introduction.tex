%Chapter of Introductions
%Pretty self explanatory really

\chapter{Introduction}
\label{intro}

%----------------------------------------------------------------------------------------

% Define some commands to keep the formatting separated from the content 
\newcommand{\keyword}[1]{\textbf{#1}}
\newcommand{\tabhead}[1]{\textbf{#1}}
\newcommand{\code}[1]{\texttt{#1}}
\newcommand{\file}[1]{\texttt{\bfseries#1}}
\newcommand{\option}[1]{\texttt{\itshape#1}}

%----------------------------------------------------------------------------------------

The Internet of Things or IoT is the concept of a huge network of physical objects connected and communicating to themselves and to the world wide web.
Devices can include domestic appliances, buildings, cars. As it becomes a rapidly growing concept with over 50 million devices expected to be connected to the web by 2020\cite{50milby2020}, the security of the transmissions of these devices is becoming a more and more pressing issue. IoT's main benefits are the remote control of devices and appliances, for the device to have the ability to send information about it's state, such as a vending machine reporting that it has run out of a certain item, and to allow the machines to be more automated and to work with other machines, like a home hub device that can turn on the lights and central heating when an occupant is arriving home, with the lights and heating not being connected to each other but to the central hub.
	
	However IoT will be ultimately be useless if it is unsecure. IoT is an emerging field but there have already been some high profile security disasters. Ranging from relatively less serious problems such as some ``hackers''  been able to glean important wifi information from your internet connected lights \cite{hackingsum} to very concerning, potentially fatal security breaches like someone gaining unauthorised access to your car and assuming control. There have been three examples of this in the last few years with a Jeep Cherokee\cite{jeephack}, a Toyota Prius\cite{priushack} and a Tesla Model S\cite{teslahack} being the cars effected. The hackers were able to control the accelerator, door locks and brakes, among other things. This highlights a very real problem that will only become more important. Too often security is an afterthought but it really needs to be built into products from the offset.

	The challenge is to provide a cryptographic solution that is similar in strength of security to solutions that are implemented on servers and computers but on much smaller and less powerful devices. A solution that performs at an acceptable speed, good security with reduced power consumption.
	
	Google and British Gas have recently released Nest and Hive respectively. Nest was released in 2014 and Hive in 2013. These both involve controlling your central heating, among other things, remotely and programming in days when you won't be at home and therefore have no need of heating. However, it was two whole years later upon independent investigators discovering that information, about dates when the heating was on because the occupants were in and off when they were away from home, was being sent unencrypted that British Gas Hive decided to encrypt their products transmissions. With the release of Hive 2 they patched the problems found but they should never have been there. Found in the same investigation, Google had a lesser fault which was sending the post code of the user unencrypted, which has since been patched. It is only when caught or there is a high profile breach that companies take the steps to full secure their customers information \cite{which}. 
	
	This is a new implementation of an old problem, developers and companies don't employ effective security of data, moving this to smaller devices doesn't change this fundamental problem. There are studies about and examples cryptographic systems for microcontrollers, it is not something that cannot be done. However, perhaps it is too difficult or complicated at present for smaller development teams or has two much of a food print in terms of computational resources and time that it gets pushed to the side. Data security is a fundamentally important concept and one that will need to implemented in every application that involves user's private data otherwise there is the potential for loss of money, intellectual property, goods, reputation and health. These problems affect both companies and the consumer.
	
	This project will look at a way to simply secure the private data of a user as it is sent across unsecure networks, the internet. The example used to work with is the secure transmission of a users private home temperature data. If they have a system that monitors the temperature of all the rooms, that data can be used to figure out when they are likely to be home or not. So, using a base platform in the user network that has access to the temperature sensors throughout the house, it takes the sensor data signs then encrypts it and sends it to a remote server. 
	
	
	741