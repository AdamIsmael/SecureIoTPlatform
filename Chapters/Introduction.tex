%Chapter of Introductions
%Pretty self explanatory really

\chapter{Introduction}
\label{intro}

%----------------------------------------------------------------------------------------

% Define some commands to keep the formatting separated from the content 
\newcommand{\keyword}[1]{\textbf{#1}}
\newcommand{\tabhead}[1]{\textbf{#1}}
\newcommand{\code}[1]{\texttt{#1}}
\newcommand{\file}[1]{\texttt{\bfseries#1}}
\newcommand{\option}[1]{\texttt{\itshape#1}}

%----------------------------------------------------------------------------------------

The Internet of Things or IoT is the concept of a huge network of physical objects connected and communicating to themselves and to the world wide web.
Devices can include domestic appliances, buildings, cars. As it becomes a rapidly growing concept with over 50 million devices expected to be connected to the web by 2020, (need ref)
the security of the transmissions of these devices is becoming a more and more pressing issue. IoT's main benefits are the remote control of devices and appliances, 
for the device to have the ability to send information about it's state, such as a vending machine reporting that it has run out of a certain item, and to allow the machines to be 
more automated and to work with other machines, like a home hub device that can turn on the lights and central heating when an occupant is arriving home, with the lights and heating not being connected to each other but to the 
central hub.
	
	However IoT will be ultimately be useless if it is unsecure. IoT is an emerging field but there have already been some high profile security disasters. Ranging from relatively less serious problems such as
some ``hackers''  been able to glean important wfif information from your internet connected lights\cite{[1]} to very concerning, potentially fatal security breaches like someone gaining unauthorised access to your car
and assuming control. There have been three examples of this with a Jeep Cherokee, Toyota Prius and Tesla. The hackers were able to control the accelerator, door locks and brakes, among other things. This highlights
a very real problem that will only become more important. Too often security is an afterthought but it really needs to be built into products from the offset.
	
Within the last three years there have been three high profile security breaches on commercial cars, one on a Cheroke Jeep \cite{jeephack}, a Toyota Prius \cite{priushack} and a tesla \cite{teslahack}

With that in mind the subject of this report is the secure transmission of a users private home temperature data. If they have a system that monitors the temperature of all the rooms, that data can be used to figure out when they
are likely to be home or not. So, using an Arduino Due as the base station that talks to the temperature sensors throughout the house, it takes the sensor data signs then encrypts it and sends it using an Ethernet Shield to a remote
server. 